%!TEX program = xelatex
% 完整编译: xelatex -> biber/bibtex -> xelatex -> xelatex
\documentclass[lang=cn,a4paper,chinesefont=founder]{shu-lab-report}


% 本文档命令
\usepackage{array}
\newcommand{\ccr}[1]{\makecell{{\color{#1}\rule{1cm}{1cm}}}}
\renewcommand{\lstlistingname}{代码}




\begin{document}

%%%%%%%%%%%%%%%%%%%%%%%%%%%%%
%% The Cover Page
%%%%%%%%%%%%%%%%%%%%%%%%%%%%%

\thispagestyle{empty}

~\\

\vspace{3cm}

\begin{figure}[!htbp]
    \centering
    \includegraphics[width=10cm]{image/shulogo.png}
\end{figure}

\centerline{\large{\textbf{SHANGHAI  UNIVERSITY}}}

\vspace{8mm}

\centerline{\kaishu\Huge{\textbf{操作系统(一)实验报告}}}

\vspace{8mm}


\vspace{16mm}

\begin{center}
\renewcommand\arraystretch{1.5}
\begin{tabular}{r c}
    \makebox[8em][s]{\LARGE{组号}}    & \LARGE{第x组}\\         \cmidrule(l){2-2} 
    \makebox[8em][s]{\LARGE{学号姓名}} & \LARGE{20122012张三}\\  \cmidrule(l){2-2} 
    \makebox[8em][s]{\LARGE{实验序号}} & \LARGE{x}\\            \cmidrule(l){2-2} 
    \makebox[8em][s]{\LARGE{日期}}    & \LARGE{2022年10月21日}\\ \cmidrule(l){2-2}
\end{tabular}
\end{center}

\vspace{8mm}

%%%%%%%%%%%%%%%%%%%%%%%%%%%%%
%% The Content
%%%%%%%%%%%%%%%%%%%%%%%%%%%%%

\newpage
\setcounter{page}{1}

\section{实验目的与要求}

\section{实验环境}

\section{实验内容及其设计与实现}

\subsection{实验内容1}

\subsubsection{子项说明}

\subsection{实验内容2}

\textbf{说明}:叙述各个实验任务的过程和结果,着重说明各个实验所反映的
操作系统设计原理,加以评述。

\section{收获与体会}

\textbf{说明}:撰写完成该实验后的收获与体会。

\section{模板测试部分(实际编写时请删除)}

如\coderef{code:example}所示,为示例代码。

\begin{lstlisting}[language=C, caption={示例代码}, label={code:example}]
#include <stdio.h>
int main() {
    printf("Hello World!");
    return 0;
}
\end{lstlisting}


如\figref{fig:example}所示,为上海大学校徽。

\begin{figure}[!htbp]
    \centering
    \includegraphics[width=0.5\textwidth]{image/shulogo.png}
    \caption{上海大学校徽}
    \label{fig:example}
\end{figure}

如\tabref{tab:example}所示,为测试表。

\begin{table}[!htbp]
    \caption{测试表}
    \label{tab:example}
    \centering
    \begin{tabular}{cc}
        \toprule
            & value\\
        \midrule
        1   & 1     \\
        2   & 2     \\
        \bottomrule
    \end{tabular}
\end{table}

\end{document}
